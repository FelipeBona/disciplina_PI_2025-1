\documentclass{article}
\author{Felipe Karmann, João Souza}
\date{30 de maio de 2025}
\title{Modelagem e Análise de Riscos Ambientais em Blumenau por Meio de Imagens de Satélite}

\begin{document}

\maketitle

\section{Descrição do problema}

A cidade de Blumenau é historicamente vulnerável a enchentes, sendo este fato devido à sua geografia montanhosa, rios urbanos e padrões de chuva intensos. A gestão de riscos naturais requer ferramentas que combinem automação, visão computacional e análise de dados geográficos para monitorar o uso do solo e identificar áreas críticas.

A partir dos artigos escolhidos para servir de base para o nosso trabalho, o problema a ser resolvido é a criação de um programa que detecte padrões e características do terreno em fotos tiradas por satélites, e, com as informações colhidas no estudo de Lacerda, Fonseca e Faria (2017), calcule as probabiliades de alagamento na região retratada. O algoritmo tem por objetivo ajudar órgãos públicos, como a Defesa Civil, na tomada de decisão rápida e baseada em evidência visual.

\section{Montagem da base}
\subsection{Fonte de dados}

A base de imagens foi criada com fotos de satélite da região de Blumenau, com foco naquelas com histórico significativo de alagamento.

\subsection{Estrutura da base}

As imagens foram rotuladas com base em coloração e características visuais, utilizando segmentação por cor em espaço HSV em vez de máscaras manuais.

\section{Preparação}

O programa usa subtração de cores para encontrar em cada imagem seções que correspondam aos quatro tipos de solo (área urbana, solo exposto, pastagem e matas), e as áreas alagadas, se houver.

Cada imagem passa pelas seguintes etapas:
\begin{itemize}
    \item conversão para HSV: o modelo trabalha no espaço HSV (Hue, Saturation, Value), mais robusto à variação de iluminação;
    \item redução de ruído com filtro bilateral;
    \item criação de máscaras binárias para classes de terreno (matas, urbana, solo exposto, pastagem e enchente).
\end{itemize}

A função \texttt{enhance\_water\_detection()} melhora a identificação de áreas alagadas, mesmo sob sombra ou baixa visibilidade.

\section{Modelo da rede}

Embora não se trate de uma rede neural convencional, este algoritmo pode ser interpretado como uma rede de regras heurísticas e de segmentação que classifica pixels da imagem com base em cores HSV.

\section{Treinamento}

Este algoritmo não requer treinamento supervisionado pois não utiliza redes neurais; ele opera por regras fixas. No entanto, as faixas HSV foram calibradas manualmente com base em amostras de imagens reais de Blumenau.

\section{Classificação e testes}

Durante a classificação, o algoritmo atribui categorias de risco às áreas da imagem com base na predominância de cada tipo:
\begin{itemize}
    \item solo exposto ou pastagem $>$ 8\% $\rightarrow$ risco de erosão ou deslizamento;
    \item enchente $>$ 3\% $\rightarrow$ risco de alagamento urbano ou rural.
\end{itemize}

Exemplo de estatísticas geradas:

\begin{verbatim}

  Estatísticas da imagem:
- Matas (Baixo risco): 38.45%
- Área Urbana (Médio risco): 23.12%
- Pastagem (Alto risco): 14.80%
- Solo Exposto (Alto risco): 12.55%
- Área Alagada (Enchente): 4.10%

  ALERTA: Areas de enchente detectadas
  ALERTA: Muito solo exposto risco de erosão

\end{verbatim}

\section{Código}

O programa está dividido em funções-chave:

\begin{itemize}
\item \texttt{load\_image(path)}: carrega e valida a imagem de entrada;

\item \texttt{create\_masks(hsv)}: gera máscaras binárias para cada classe de risco com base em faixas de cor HSV;

\item \texttt{enhance\_water\_detection(hsv)}: aprimora a detecção de água utilizando múltiplas técnicas (tons de azul, marrom e sombras);

\item \texttt{apply\_masks(image, masks)}: aplica as máscaras sobre a imagem com cores indicativas e transparência;

\item \texttt{calculate\_risk\_areas()}: calcula a porcentagem de cada classe na imagem;

\item \texttt{add\_legend\_and\_stats()}: gera legenda visual e insere alertas diretamente na imagem final;

\item \texttt{process\_image(image\_path)}: integra todas as etapas para processar uma imagem completa.
\end{itemize}

\section{Resultados}
\subsection{Demonstração de entradas e saídas}

Entrada: imagem de satélite da cidade de Blumenau após período de chuvas:

Saída: imagem processada com destaque visual das áreas e legendas:

\begin{itemize}
    \item verde escuro $\rightarrow$ matas;
    \item vermelho $\rightarrow$ solo exposto;
    \item azul $\rightarrow$ áreas urbanas;
    \item roxo ou preto $\rightarrow$ enchente.
\end{itemize}

\end{document}
